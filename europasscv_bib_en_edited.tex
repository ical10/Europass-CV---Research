% !TEX encoding = UTF-8
% !TEX program = pdflatex
% !TEX spellcheck = en_GB

\documentclass[english,a4paper]{europasscv}
\usepackage[english]{babel}

\usepackage[backend=biber,autolang=hyphen,sorting=none,style=numeric,maxbibnames=99,doi=false,isbn=false,maxcitenames=3]{biblatex}
\usepackage{csquotes}
\usepackage{europasscv-bibliography}
\usepackage{siunitx}

\bibliography{europasscv_example}
% in the bibliography, mark all occurrences in bold:
\ecvbibhighlight{Rizal}{Mochammad-Husni}{MH}

\ecvname{Mochammad Husni Rizal}
\ecvaddress{Amarapura Blok B-3/31, 15313 Tangerang Selatan (Indonesia)}
\ecvmobile{+33 6 59 60 93 00}
%\ecvtelephone{+353 127 6689}
%\ecvworkphone{+353 999 888 777}
\ecvemail{mochhusnir@gmail.com rizal@ipgp.fr}
\ecvhomepage{https://ical10.github.io/}
\ecvgithubpage{https://github.com/ical10?tab=repositories}
% \ecvgitlabpage{www.gitlab.com/smith}
\ecvlinkedinpage{https://www.linkedin.com/in/mochammad-husni-rizal-56311847/}
% \ecvorcid[label, link]{0000-0000-0000-0000}
%\ecvim{AOL Messenger}{katie.smith}
%\ecvim{Google Talk}{ksmith}
\ecvim{Skype}{mochhusnir}


% \ecvgender{Female}
\ecvdateofbirth{10 July 1992}
\ecvnationality{Indonesian}

\ecvpicture[width=3.8cm]{Pas_Foto_cropped.jpeg}

% \date{}

\begin{document}
  \begin{europasscv}

  \ecvpersonalinfo
  \ecvbigitem{About me}{Holds a Master in Earth and Planetary Sciences, and Environment from Université de Paris. Aspiring to be a researcher who specialised in unravelling subsurface volcanic processes using seismology. Currently studying the current volcanic plumbing system beneath Mt. Merapi, Java Island, Indonesia, using body-wave tomography. Interested as well in analysing seismic noise in a volcanic region, mainly to explain the relationship between the propagation velocity variations and the opening or closing of fractures.}
  
    \ecvsection{Education and training}
  
  \ecvtitle{Sept 2018 -- July 2020}{Master in Earth Science}
  \ecvitem{}{Université de Paris}
  \ecvitem{}{
      \begin{ecvitemize}
	\item Discipline: Geophysics; predicate: with high honors; results: 14.82/20 
	\item Final year internship subject: Structure of Merapi-Merbabu complex, Central Java, Indonesia, modeled from body wave tomography
	\item the results from tomographic inversion highlight: 1) a zone where fluid migration occurs during the $2018$ phreatic eruption,
	2) a hot magmatic body which "goes around" a much cooler, degassed, older magma deposits at a depth of around \SI{2}{\km} below the summit
     \end{ecvitemize}
  }

  \ecvtitlelevel{1 Sept 2010 – 28 Jan 2015}{Bachelor of Science in Geophysics}{}
  \ecvitem{}{Universitas Gadjah Mada, Yogyakarta (Indonesia)}
  \ecvitem{}{  
  \begin{ecvitemize}
      \item Predicate: Cum laude; cumulative GPA: 3.62 / 4.00
      \item Subject of thesis: Application of Double-Difference Method to Relocate Microearthquake Hypocenter in “KRHR” Geothermal Field
      \item The double-difference method applied at this geothermal field is capable to provide a better hypocenter solution, where the relocated hypocenters move toward a certain structure and form a clear pattern of seismicity
  \end{ecvitemize}}
  
    \ecvsection{Work experience}
  
  \ecvtitle{Feb 2019 - June 2019}{
Research internee
}
  \ecvitem{}{
Institut de Physique du Globe de Paris (IPGP), Paris (France)
}
\ecvitem{}{
\begin{ecvitemize}
      \item Subject: Structure of Merapi-Merbabu complex, Central Java, Indonesia, modeled from body wave tomography; Supervisors: Jean-Philippe M\'etaxian, Ivan Koulakov
      \item Note obtained: 16/20 
      \item Main results: Highlight of the zone where fluid migration occurs during the 2018 phreatic eruption (from \SI{900}{\meter} to \SI{1800}{\meter} above sea level), and a suggestion of a hot magmatic body "going around" a much cooler, degassed, older magma deposits at a depth of around 2 km below the summit.
  \end{ecvitemize}
  }
  
  \ecvtitle{March 2019 - May 2019}{
Research internee
}
  \ecvitem{}{
Institut de Physique du Globe de Paris (IPGP), Paris (France)
}
\ecvitem{}{
\begin{ecvitemize}
      \item Subject: characterisation of eruptive activity of Mount Ibu (Indonesia) observed by a seismic array and an infrasound sensor; Supervisor: Jean-Philippe M\'etaxian
      \item Note obtained: 16.25/20 
      \item Main results: A combination of a seismic antenna and an infrasound sensor was able to identify explosive events and to delimit the location of the source of explosions and deeper events at the crater area.
  \end{ecvitemize}
  }
  
  \ecvtitle{15 Aug 2015 – 16 Aug 2018}{
Technician of geophysical equipments
}
  \ecvitem{}{
French National Research Institute for Sustainable Development (IRD), Jakarta (Indonesia)
}
  \ecvitem{}{  
  \begin{ecvitemize}
      \item maintain geophysical equipments for DOMERAPI/VELI project at Mt. Merapi volcano in Indonesia
      \item fieldworks in other parts of Indonesia, such as on: Mt. Dukono and Mt. Gamalama volcano, Opak fault - Yogyakarta, and Muara Labuh geothermal field - West Sumatra
  \end{ecvitemize}}
   
  \ecvtitle{4 Aug 2014 – 4 Oct 2014}{Trainee}
  \ecvitem{}{Star Energy Geothermal Wayang Windu, Jakarta (Indonesia)}
  \ecvitem{}{
  Study of hypocenters relocation in a geothermal field by using double-difference method
  }
  \ecvitem{}{
  Main results: The double-difference method gives better hypocentral solutions, with the relocated hypocenters shifted toward a specific structure and formed a clear seismicity pattern.
  This pattern is visible at the northern part of the geothermal field with N\ang{345}E strike, \ang{80} eastward dip, and it is distributed between $0$ and \SI{800}{\meter} deep.}
  
  \ecvtitle{1 Apr 2014 – 31 Apr 2014}{Trainee}
  \ecvitem{}{Geothermal Department, Center for Geological Resources, Ministry of Energy and Mineral Resources of the Republic of Indonesia, Bandung (Indonesia)}
  \ecvitem{}{Study of geomagnetic data processing and simple interpretation in a geothermal field}
  
  \nocite{*}
  \renewcommand{\section}[2]{\ecvsection{#2}}
  \printbibtabular[title=Publications]
  
    \ecvsection{Personal skills}
  \ecvmothertongue{Indonesian}
  \ecvlanguageheader
  \ecvlanguage{English}{C1}{C1}{C1}{C1}{C1}
  \ecvlanguagecertificate{IELTS (date of issue: 3 March 2018). Overall band score: 7.5/9.0 (C1)}
  \ecvlanguage{French}{B2}{B2}{B2}{B2}{B2}
  \ecvlanguagecertificate{Diplôme d'études en langue française (DELF) B2}
  \ecvlanguagefooter
  
  \ecvblueitem{Programming skills}{
  \begin{ecvitemize}
    \item All of the competences below had been obtained mostly through my research training at Institut de Physique du Globe de Paris
    \item Independent competence to work with several programming languages, such as: \textbf{bash, Python, MATLAB, and SQL}
    \item Able to understand how script written in different languages works, and can comprehend various concepts and implementation in programming such as: Object-oriented Programming, data structure, signal processing, text manipulation, numerical modeling
  \end{ecvitemize}
  }
   
  \ecvblueitem{Technical skills}{
  \begin{ecvitemize}
    \item Experienced in installation, usage and maintenance of geophysical instruments related to volcanological research
    \item Basic knowledge of computer networking concepts: telemetry system, network configuration, and other concepts related to autonomous geophysical monitoring station design
    \end{ecvitemize}
    }
  
  \ecvsection{Grants and scholarships}
    \ecvblueitem{2019 - 2020}{Master's excellence grants of Make Our Planet Great Again, an initiative launched by France's president Emmanuel Macron}
    \ecvblueitem{2015}{Travel grants by European Association of Geoscientists and Engineers (EAGE) to attend the $77^{th}$ EAGE Conference and Exhibition in Madrid, Spain}
    \ecvblueitem{2010, 2011, and 2013}{Students’ Achievement Scholarship (Peningkatan Prestasi Akademik in Indonesian), awarded by Ministry of Research, Technology and Higher Education of the Republic of Indonesia}

\end{europasscv}
  
\end{document}
